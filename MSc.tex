\documentclass[a4paper,12pt,twoside,BCOR=10mm]{scrbook}

% UoI MSc thesis template (English) V2.0.4 25.4.2022

% The license of this template is not fully clear, but: This template
% version is based on an earlier LaTeX template that was once available
% on a SENS UGLA page where the author and license are unknown.  As
% that version has obviously been put once online with the intend to be
% used by students, providing and using it should not be problem.

% Helmut Neukirchen https://uni.hi.is/helmut updated that earlier
% template using a tikzpicture-based approach for the title page
% created by Þór Arnar Curtis.

% The included logos are very likely copyrighted by the University of
% Iceland, but
% https://zeroheight.com/1b323cfd9/p/20f1e6-theses/b/29a61a
% states that the document is intended to make it easier for
% students, staff, and print-shops to format the thesis and ensure a
% standardised appearance. So providing them here and using them
% should be legal.


% Search below for "MODIFY THESE LINES ONLY": there you can enter your name, thesis title, etc.

% BibLaTeX is assumed for references. While Overleaf does this
% automatically for you, if you run it locally on the commandline,
% then you need to run first pdflatex MSc.tex, then biber MSc (without
% the .tex extension), and then again pdflatex MSc.tex


% Packages
% BETRI TÖFLUR OG LISTAR
\usepackage{booktabs}         % Fyrir betri töflur
\usepackage{caption}          % M.a. aukabil á undan töflufyrirsögnum
\captionsetup{font=small,   % Smærra letur á mynda- og töflutexta,
              labelfont=bf, % "Mynd 1" og "Tafla 2" feitletrað 
              margin=2cm,
              skip = 0.5\baselineskip} 
\usepackage[utf8]{inputenc}
\usepackage[icelandic, english]{babel}
\usepackage{t1enc}
\usepackage{graphicx}
\usepackage[intoc]{nomencl}
\usepackage{enumerate,color}
\usepackage{url}
\usepackage[pdfborder={0 0 0}]{hyperref}
\BeforeTOCHead[toc]{\cleardoublepage\pdfbookmark{\contentsname}{toc}} % Add Table of Contents to PDF "bookmark" table of contents
\usepackage{appendix}
\usepackage{eso-pic}
\usepackage{amsmath}
\usepackage{amssymb}
\usepackage{subcaption}
%\usepackage[sf,normalsize]{subfigure}
\usepackage[format=plain,labelformat=simple,labelsep=colon]{caption}
\usepackage{placeins}
%\usepackage{tabularx}
\usepackage{enumitem}
\usepackage{listings}
\usepackage{xspace}
\usepackage{float}
\usepackage{caption}
% Packages used for title page layout
\usepackage{xcolor}
\usepackage{tikz}
\usepackage{arydshln}
\usepackage{acronym}
\usetikzlibrary{positioning}
\pgfkeys{/pgf/number format/.cd,fixed,precision=2}

% https://tex.stackexchange.com/questions/374661/keyword-highlighting-not-working-in-python-with-listings
\lstdefinestyle{Python}{
    language        = Python,
    basicstyle      = \ttfamily,
    keywordstyle    = \color{blue},
    keywordstyle    = [2] \color{teal}, % just to check that it works
    stringstyle     = \color{green},
    commentstyle    = \color{red}\ttfamily
}

\newcommand{\averageWindSpeedLimit}{25\xspace}
\newcommand{\nHiddenLayers}{6\xspace}
\newcommand{\nPCA}{10\xspace}
\newcommand{\nunits}{256\xspace}
\newcommand{\nepochs}{200\xspace}
\newcommand{\batchsize}{128\xspace}
\newcommand{\penaltyRegularization}{0.01\xspace}
\newcommand{\activation}{relu\xspace}
\newcommand{\startDateVedur}{2004\xspace}

% Blue color according to HÍ corporate design
\convertcolorspec{RGB}{16,9,159}{rgb}\tmphiblue
\definecolor{hiblue}{rgb}\tmphiblue


\setlength{\parskip}{\baselineskip}
\setlength{\parindent}{0cm}
\raggedbottom

\setkomafont{captionlabel}{\itshape}
\setkomafont{caption}{\itshape}
\setkomafont{section}{\FloatBarrier\Large}
\setcapwidth{\textwidth}
%\setcapwidth[l]{\textwidth} % The original template had the [l] which leads to a warning that it gets ignored, so to reduce warnings, removed it.
\setcapindent{1em}


\usepackage{lmodern} % Use Latin Modern (instead of the default Computer Modern that is rendered using a bitmap font).
\usepackage{fixcmex} % To fix that Latin Modern large symbol math fonts has by default only one size: https://tex.stackexchange.com/a/621536
% Times new roman font instead of the standard LaTeX fonts: has not been test -- try this on your own risk
%\usepackage[T1]{fontenc}
%\usepackage{mathptmx}

%%%%%%%%%%%%%%%%% Configurations (Useful defaults, but OK to change %%%%%%%%%%%%%%%%%%%
\graphicspath{{figs/}} % Figures in directory figs

% Bibliography

% \usepackage[sort&compress,authoryear]{natbib} % Uncoment if you want to used NatBib instead of BibLaTeX (and comment the bitlatex line below)

\usepackage{biblatex}  % BibLaTeX used for references. 
\usepackage{csquotes} % BibLaTex wants to have context sensitive quotes
\addbibresource{references.bib} %  Name of *.bib file containing references


%%%%%%%%%%% MODIFY THESE LINES ONLY %%%%%%%%%%%%%%%%%%%%%%%%%%%%%%%%%%%%%%%%%%%%%%%%%%%%%%%%%

% Some note on advisor(s) vs. thesis committee: At the School of
% Engineering and Natural Sciences, according to Regulation
% no. 994-2017
% https://english.hi.is/university_school_of_engineering_and_natural_sciences/regulation_no_994_2017_on_masters_study_at
% Article 5., a student has an administrative supervisor who is
% typically also the academic supervisor, i.e. supervising the thesis.
% If someone from outside is supervising, than the person becomes the
% academic supervisor and you have in addition the administrative
% supervisor from within HÍ.
% In addition, according to Article 7., there is a Master's degree
% committee that includes at least two persons, one of which shall be
% the student’s administrative supervisor.
% Despite these formally defined roles, it is somewhat a matter of
% taste whether you list all persons as supervisors or all as just
% thesis committee or list the one single academic supervisor as
% supervisor and then all persons (repeating the supervisor's name)
% as thesis committee.
% In the settings, that follow here, you can set this and many other settings.

\def\thesisyear{2025}       					% Year thesis submitted
\def\thesismonth{June}					% Month thesis submitted, e.g. "December"
\def\thesisauthor{Brynjar Geir Sigurðsson}				% Thesis authoreiningaraðferðinni
\def\thesistitle{A Neural Network Approach to Predicting Gust Factors in Complex Landscape} % Title of thesis
%\def\thesissubtitle{XXSubtitle is rarely usedXX}		% Subtitle of thesis (optional)
\def\thesisshorttitle{} 	% Optional: if title of thesis is longer than 50 characters, it would not fit on the spine (kjölur) and in this case you, need to provide a short title here for the print shop. Otherwhise: make it empty
\def\thesiscredits{60} 						% Credits awarded for the project
\def\thesissubject{Mechanical Engineering}
\def\thesiskind{M.Sc.}					% E.g. M.Sc. or Ph.D. thesis
\def\thesiskindformal{CCMagister Scientiarum}			% E.g. Magister Scientiarum
\def\thesisschool{School of Engineering and {Natural Sciences}}		% School
\def\thesisfaculty{Industrial Engineering, Mechanical Engineering and Computer Science}% Faculty name without "Faculty of" (gets added) 
\def\thesisaddress{Dunhagi 5}			        % Faculty or school office address street
\def\thesispostalcode{107}			                % Faculty or school office zip code
\def\thesistelephone{525 4000}					% Office telephone
%\def\thesispublisher{XX}					% Publisher (not used)
\def\thesissupervisors{Kristján Jónasson}					% Names of supervisors (split by \\ if more than one).
\def\thesisnrofsupervisors{1}					% Number of supervisors (to use "Supervisor" vs. "Supervisors")
\def\thesiscommittee{Kristján Jónasson \\ Ólafur Pétur Pálsson \\ Guðrún Nína Petersen}			% Thesis commitee must include the supervisor(s)
\def\thesisexaminer{XXNN3XX}				        % Examiner (prófdómari)
\def\thesisISBN{}           					% Thesis ISBN number (keep empty: not used anymore)
\def\thesisprinting{}						% Name of printsthop (keep empty if thesis is never printed)
%\def\thesisprinting{Háskólaprent, Fálkagata 2, 107 Reykjavík}	% Name of printsthop (keep empty if thesis is never printed)
\def\thesislicense{This thesis may not be copied in any form without author permission.} % Set license here (could also be some Creative Commons license)
%\def\thesiskeywords{Keyword1, Keyword2, Keyword3}		% Keywords (not used anywhere, hence commented out)

%%%%%%%%%%% STOP MODIFYING HERE %%%%%%%%%%%%%%%%%%%%%%%%%%%%%%%%%%%%%%%%%%%%%%%%%%%%%%%%%

%%%%%%%%%%% Next modifications: search for "START MODIFYING HERE AGAIN" below %%%%%%%%%%

% We need this command if someone used \\ in the thesis title
\newcommand{\removelinebreaks}[1]{%
  \begingroup\def\\{}#1\endgroup}

% To be able to use units in equations. From https://qerub.se/typesetting-units-in-latex
\newcommand{\unit}[1]{\ensuremath{\, \mathrm{#1}}}

\begin{document}
\hypersetup{pageanchor=false}
\pagenumbering{Alph} % To prevent page numer "1" to be used multiple times, used "A", "B", etc. for the first pages
\begin{titlepage} % This titlepage environment spans in fact multiple pages

% This must be placed after "\begin{document}":
\lstset{
    frame       = single,
    numbers     = left,
    showspaces  = false,
    showstringspaces    = false,
    captionpos  = t,
    caption     = \lstname
}

% This is the cover title page. If you go to a print shop, they will
% ignore it and create their own cover page, i.e. this cover page here
% is only used by the PDF version that gets electronically archived.

  \thispagestyle{empty}
  
  % The banner at top and bottom (using a tikz overlay)
  \begin{tikzpicture}[remember picture,overlay]
    \node[anchor=north west, inner sep=0pt] at (current page.north west)
        {\includegraphics[width=\paperwidth]{banner}}; % The top banner (as a PNG) % TODO: A vector graphic would be better

    \node(bottom)[shape=rectangle, fill=hiblue, minimum height=10mm, minimum width=\paperwidth, anchor=south west] at (current page.south west) {}; % The bottom banner (a filled rectangle)

    \node[above=0.4cm of bottom] {
        \begin{tabular}{c} 
          \sffamily \small \textcolor{hiblue}{\textbf{\MakeUppercase{Faculty of \thesisfaculty{}}}}
        \end{tabular}
    };
  \end{tikzpicture}

  \enlargethispage{3cm}
  \vspace*{3.5cm} % Here starts the white space below the top banner
  
  % The centering used below is with respect to the page margins which
  % are not the same on left and right which prevents proper centering
  % with respect to the tikz centering (and the title page in
  % general).  Instead, we use a minipage that we shift horizontally
  % by 2.6 cm.  But minipage sets \parskip to 0, so we need to save
  % and restore it.  To be able to use vfill/stretch in a minipage,
  % the height needs to be specified: 20.0 cm.
  \newlength{\currentparskip}
  \setlength{\currentparskip}{\parskip}
  \hspace*{-2.6cm}
  \begin{minipage}[t][20.0cm]{1.0\paperwidth}
    \setlength{\parskip}{\currentparskip}
    \begin{center}
      \vspace*{ \stretch{1.5} }
      \huge \sffamily \bfseries \thesistitle{}
    
      %\normalfont \LARGE \sffamily \thesissubtitle{}

      \vspace{ \stretch{1.0} }
      \normalfont \Large \sffamily \thesisauthor{}

      \vspace*{ \stretch{2.75} }

      \thesismonth{}~\thesisyear{} % E.g. "March 2022"

      \vspace{ \stretch{0.5} }
    
      \normalfont \Large \sffamily {\thesiskind{}~thesis \\
      in \thesissubject{}}

      \vspace*{ \stretch{1.0} }
    \end{center}
  \end{minipage}

  \newpage

  \thispagestyle{empty} \mbox{} % This is the inside page of the cover (cover verso) which remains empty

  \newpage


  \thispagestyle{empty} % This is the inner title page
  \begin{center}
    \vspace*{ \stretch{0.5} }

    \Large \sffamily \bfseries \thesistitle{}
    
    %\normalfont \large \sffamily \thesissubtitle{}

    \vspace*{ \stretch{1.0} }

    \sffamily{\thesisauthor{}}
    
    \vspace*{ \stretch{1.0} }
    \normalsize \thesiscredits{}~ECTS thesis submitted in partial fulfillment of a \\
    \textit{\thesiskindformal{}} degree in \thesissubject{}
    \large
    
    \ifx\thesissupervisors\empty % Only print supervisor part if supervisor names are not empty
    \else
      \vspace*{ \stretch{1.0} }
      \ifnum\thesisnrofsupervisors>1 Supervisors \\
      \else Supervisor \\
      \fi
      \thesissupervisors{}
    \fi  

    \ifx\thesiscommittee\empty % Only print thesis committee part if committee names are not empty
    \else
      \vspace*{ \stretch{0.25} }
      \thesiskind{}~Committee\\
      \thesiscommittee{}
    \fi
      
    \ifx\thesisexaminer\empty % Only print examiner part if thesisexaminer is not empty
    \else
      \vspace*{ \stretch{0.25} }
      Examiner \\
      \thesisexaminer
    \fi
      
    \vspace*{ \stretch{1.0} }

    Faculty of \thesisfaculty \\
    \thesisschool \\
    University of Iceland \\
    Reykjavik, \thesismonth~\thesisyear

    \vspace*{ \stretch{0.5} }
  \end{center}

  \newpage

  \thispagestyle{empty} % This is the title verso (colophon), i.e. imprint/copyright page
  \vspace*{\fill}
  % \setcounter{page}{0} \renewcommand{\baselinestretch}{1.5}\normalsize
  \sffamily{\removelinebreaks{\thesistitle}} \\
  \ifx\thesisshorttitle\empty % Show only if shorttitle is provided
  \else
  (\sffamily{\thesisshorttitle{}}) \\
  \fi
  %\sffamily{\removelinebreaks{\thesissubtitle{}}} \\

    
  \thesiscredits ~ECTS thesis submitted in partial fulfillment of a \thesiskind{}~degree in \thesissubject
\\ \\
  Faculty of \thesisfaculty \\
  \thesisschool \\
  University of Iceland \\
  \thesisaddress \\ 
  \thesispostalcode, Reykjavik 
  Iceland

  Telephone: \thesistelephone \\ \\ 
  \vspace*{\lineskip}

  Bibliographic information: \\
  \thesisauthor{} (\thesisyear{}) \emph{\removelinebreaks{\thesistitle{}}}, \thesiskind{}~thesis, Faculty of \thesisfaculty, University of Iceland.\\

  Copyright \textcopyright~\thesisyear~ \thesisauthor \\
  \thesislicense{}\\

  \ifx\thesisISBN\empty % Show only if ISBN is provided
  \else
  ISBN~\thesisISBN
  \fi
  
  \ifx\thesisprinting\empty % Show only if print shop is provided
  \else
  Printing: \thesisprinting \\
  \fi


  Reykjavik, Iceland, \thesismonth~\thesisyear \\
  
%%%%%%%%%%% START MODIFYING HERE AGAIN %%%%%%%%%%%%%%%%%%%%%%%%%%%%%%%%%%%%%%%%%%%%%%%%%%%%%%%%%

  \newpage % Dedication page: remove completely if you have no dedication

  \thispagestyle{empty} \mbox{}

  \vfill

  \begin{center}
    \textit{
      To all the students who made the wise decision to use \LaTeX. % Replace this by your dedication
    }
  \end{center} \vspace*{5cm}

  \vfill 

%%%%%%%%%%%%%%%%%%%%%%%%% If you have no dedication: remove until here %%%%%%%%%%%%%%%%%%%%

\end{titlepage}

\cleardoublepage

\pagenumbering{roman} % Abstract page 

\setcounter{page}{5}

\setkomafont{section}{\huge} % The title "Abstract" and "Útdráttur" should look like the chapters, i.e. use \huge (\chapter cannot be used as this would create a new page)

\section*{Abstract}
In many industries, whether it be windmill farming or transportation, being able to precisely predict the load caused by weather is crucial to prevent harm coming to people and machinery as well as increasing productivity. The efficiency of wind turbines in high wind speed conditions is a vital challenge in optimizing renewable energy systems. Gust predictions can be instrumental in loss prevention. If looking only at wind, the most destructive moments are gusts, by definition. To predict these values there are two ways, traditional numerical weather prediction systems and machine learning or other pattern recognition systems. Computer aided numerical weather prediction systems originated in the 1950's, while the use of machine learning in numerical weather predictions is much more recent, having truly gained traction in the last decade. The goal of this thesis is try to combine these two methods to try and predict wind gust factor, where the gust factor is defined as the ratio between wind gust and average wind speed. This is done by having the input data to the machine learning model be reanalysis variables that were generated by traditional numerical weather prediction systems. In addition to these variables and derived variables, a digital elevation model (DEM) data was used. A deep neural network was constructed using some set of variables and the results are compared to a baseline as well as each other so as to show the impact each feature has.
\vfill \vspace*{1cm}

\section*{Útdráttur}
Í mörgum iðnuðum, hvort sem það er vindmyllubúgarður eða flutningar er getan til þess að spá fyrir um álag vegna veðurs mikilvæg til að koma í veg fyrir tjóns á vélum og byggingum ásamt því að koma í veg fyrir skaða á fólki. Einnig er hægt að auka framleiðni. Nýtni vindtúrbínna í hávinda aðstæðum er mikilvæg áskorun í bestun endurnýjanlega orkukerfisins. Hviðuspár geta verið stór þáttur í tjónaminnkun. Ef einungis er horft á vind, þá eru hviður hæsti álagspunktur út frá skilgreiningu. Sögulega hefur verið notast við hefðbundin spálíkön sem reiða sig á töluleg eðlisfræðileg líkön til að spá fyrir um niðurstöður. Á síðustu árum hefur gervigreind þróast mikið og líkön verið þjálfuð til þess að giska á veður. Svo miklar framfarir hafa orðið að sum líkön geta keppt við hefðbundin veðurlíkön. Markmið þessa verkefnis er reyna að nýta bæði hefðbundin veðurlíkön og gervigreind til þess að giska á hviðustuðla, þar sem hviðustuðullinn er skilgreindur sem hlutfall hæstu hviðu og meðalvinds. Þetta er gert með því að nota endurgreiningargögn úr hefðbundnu veðurlíkani sem grunngögn í gervigreindarlíkan. Einnig eru notað hæðarlíkan til að lýsa umhverfi. Djúpt tauganet var búið til með því að velja breytur úr endurgreiningargögnunum ásamt hæðarpunktum og afleiddum breytum. Þetta líkan er svo þjálfað og niðurstöður bornar saman við grunnlíkan sem og önnur líkön með öðrum breytum til að skoða áhrif breyta.
\vfill

\newpage % Empty page
\setkomafont{section}{\FloatBarrier\Large}
% The first "chapter" which will start on a new page
% Table of contents starts automatically on a right-hand side page ("recto").
% Table of contents, list of figures and tables are automatically generated by the commands below
\hypersetup{pageanchor=true}
\tableofcontents
\listoffigures
\listoftables
\lstlistoflistings

\chapter*{Abbreviations}
\addcontentsline{toc}{chapter}{Abbreviations}
\markboth{Abbreviations}{Abbreviations}
\begin{acronym}[GeoTIFF]
  \acro{API}{Application Programming Interface}
  \acro{ASL}{Above Sea Level}
  \acro{AWS}{Automatic Weather Stations}
  \acro{AWSL}{Average Wind Speed Limit}
  \acro{CARRA}{Copernicus Artic Regional ReAnalysis dataset}
  \acro{CNN}{Convolutional Neural Networks}
  \acro{DEM}{Digital Elevaiton Model}
  \acro{ELI5}{Explain Like I am 5}
  \acro{ECMWF}{European Centre for Medium-Range Weather Forecast}
  \acro{GCM}{General Circulation Model}
  \acro{GeoTIFF}{Georeferenced TIFF}
  \acro{GPU}{Graphical Processing Unit}
  \acro{HRES}{High Resolution forecas}
  \acro{IMO}{Icelandic Meteoroligcal Office}
  \acro{IRCA}{Icelandic Road and Coastal Administration}
  \acro{JNWPU}{Joint Numerical Weather Prediction Unit}
  \acro{LWM}{Large AI Weather forecast Model}
  \acro{MAE}{Mean Asbolute Error}
  \acro{MAPE}{Mean Absolute Percentage Error}
  \acro{NN}{Neural Network}
  \acro{NWP}{Numerical Weather Prediction}
  \acro{SENS}{School of Engineering and Natural Sciences}
  \acro{TIFF}{Tag Image File Format}
  \acro{UoI}{University of Iceland}
\end{acronym}

\chapter*{Acknowledgments}
\addcontentsline{toc}{chapter}{Acknowledgments}
Special thanks go to my advisor, Kristján Jónasson. He would help me with the actual work of the thesis and sit with go over the code with me. Thanks also go to Ólafur Pétur Pálsson, who would read over the thesis with notes and give helpful suggestions. Guðrún Nína Petersen, also provided valuable insight whenever questioned about meteoroligcal matters as well as giving notes on the thesis.

Finally, I would like to thank my parents without whom I would probably never had any want to finish.

\pagenumbering{arabic}
\setcounter{page}{1}

% Chapter 1

\chapter{Introduction} % Main chapter title

\label{Chapter1} % For referencing the chapter elsewhere, use \ref{Chapter1} 

%----------------------------------------------------------------------------------------

% Define some commands to keep the formatting separated from the content 
\newcommand{\keyword}[1]{\textbf{#1}}
\newcommand{\tabhead}[1]{\textbf{#1}}
\newcommand{\code}[1]{\texttt{#1}}
\newcommand{\file}[1]{\texttt{\bfseries#1}}
\newcommand{\option}[1]{\texttt{\itshape#1}}

%----------------------------------------------------------------------------------------
Wind gusts are brief increase in wind speed (lasting seconds) as compared to mean wind speed. The gust factor is defined as the peak gust divided by the mean wind speed over some defined time period. The peak wind gust is often defined as the highest 3 second rolling average measured wind speed over a period of 10 minutes, while the mean wind is the average of all measurements in the 10 minute intervalf. This thesis uses this definition. This varies, with the US using a 1 minute interval, leading to 14\% higher results \cite{why_wind_gusts}. The Navier Stokes Equation (\ref{eqn:navierstokes}) shows that the change of the wind, in time and space, is dependent upon the pressure gradient, the oscillating force of the earth (the Coriolis force), and frictional force.\cite{uncertainties_in_numerical_weather_predictions}
\begin{equation}
    \label{eqn:navierstokes}
    \frac{\delta \mathbf{V}}{\delta t} + \mathbf{V}\cdot\nabla\mathbf{V} = \underbrace{-\frac{1}{\rho}\nabla P}_{pressure} -\overbrace{ f\mathbf{k}x\mathbf{V}}^{oscillation} - g - \underbrace{\frac{\delta(u'\omega')}{\delta z} - \frac{\delta(v'\omega')}{\delta z}}_{resistance}
\end{equation}

Traditionally, numerical weather prediction (NWP) systems are used to forecast and analyze weather patterns\cite{medium_range_3d_weather_forecasting_NN}. These models describe the transition between discretized packages of atmospheric states using partial differential equations based on physical reality. These results are usually published every hour, or at courser time intervals for climate simulations. With increasing computer power and efficiency the trend is to output data more often\cite{GNP_vidtal}.They describe the state over the period and so do not necessarily grasp fluctuations well. These fluctuations would include fluctuations in the wind speed, wind gusts\cite{canNNBeatNWP}.

This thesis looks at how best to predict gust factor based on various factors, using several different data sources, including NWP and observations. Being able to accurately predict the wind gust is important as it is often the peak wind gusts that will cause failures in structures. A problem that will become increasingly prevalent in the near future\cite{nasa_extreme_weather}.

\section{Background}
The history of numerical weather predictions goes all the way back to the 1920's when Lewis Fry Richardson pioneered the field and tried to produce forecasts. The results were flawed due to noise in the calculations. ENIAC was built in 1945, it was a general purpose computer that was used, among other things, to make predictions. These predictions took 24 hours to make and were predicting 24 hours into the future. It was a proof of concept but not usable\cite{TheENIACForecastsARecreation}. In the 1950's, with the advent of computers the first operational forecasts emerged. In September of 1954, Rossby and his Stockholm based team produced the first real-time barotropic forecasts. The next year the Joint Numerical Weather Prediction Unit (JNWPU), based in Princeton New Jersey, released their first forecasts. These forecasts were for 36 hours at 400, 700 and 900 mb. The results were inferior to subjective human-based forecasts but showed that such forecasts were feasible and promoted further development in the area \cite{historyNWP}. The field of NWP has taken great strides since then following the development of computer power and efficiency.

In the last decade there has been another transformation in the field of weather prediction driven by artificial intelligence. Interest in AI has come in waves. Some progress is made, then interest dwindles. Interest in AI has been increasing steadily since 2010. Notable work that has driven this wave of interest include increase in computational abilities due to parallel processing in graphical processing units (GPU), convolutional neural networks (CNN), which allowed much faster processing of massive (image) datasets and the availability of large datasets online. It is to be noted that images are grid data with some number of channels. Using CNNs could work on any gridded data where there are some spatial features\cite{canNNBeatNWP}. Since 2018, there has been significant work done in the weather prediction field using AI. In 2018, Dueben and Bauer showed that you can build a NN that can outperform a simple persistence forecast and is competitive with very coarse-resolution atmosphere models of similar complexity for short lead times\cite{dueben2018}. Also in 2018, Scher created a deep convolutional neural network (CNN) to emulate a general circulation model (GCM, a numerical model representing the physical processes), training on the GCM which allows it to emulate the dynamics of the model and maintain stability for much longer than Dueben\cite{scher2018}. These two papers were more proof of concept rather than production ready models to replace NWP. They showed that models based on deep learning might, with further development, compete with standard models in the field.

In the last two years there have been even more developments with the emergence of Large AI Weather forecast Models (LWM). In 2024, Ling et al.\cite{SecondRevolution} tried to standardize the definition of LWM in meteorology and came up with 3 rules that need to be met to count as LWM.

\begin{enumerate}[label = Rule \arabic*:]
    \item Large Parameter Count. The number of parameters can vary wildly but a general range might be from tens of millions to billions of parameters
    \item Large Number of Predictands: predicting on different levels (such as pressure levels or height levels) and offering detailed information on the atmospheric vertical structure and surface conditions
    \item Scalability and downstream applicability. This might crystallize in predicting cyclones. Often, the teams responsible for creating these models try to show their applicability to predict cyclones when not trained specifically on cyclone data (e.g. \href{https://www.youtube.com/watch?v=PD1v5PCJs_o&ab_channel=GregBronevetsky}{GraphCast})\cite{SecondRevolution}. This is done to show the versatility of the models.
\end{enumerate}

Before 2022, LWM had been shown to be able to compete with traditional NWP for some specifc cases as well as making predictions quicker, after training. No model had shown that it could in any way completely replace the traditional systems. In early 2022, Pathak et al.\cite{FourCastNet} presented FourCastNet. FourCastNet uses an Adaptive Fourier Neural Operator model that leverages transfomer architecture rather than the popular convolutional model architechture. FourCastNet matches the performance of standard forecasting techniques at short lead times for large-scale variables and outperforms for smaller variables. It generates a week-long forecast in less than 2 seconds, orders of magnitude faster than standard physical methods\cite{FourCastNet}. In 2022, machine learning methods were presented that made predictions much faster than traditional NWP, after a one time training (or at least training that wouldn't have to be redone often). These were in some cases performing better than NWP. In 2023, Remi Lam and the GraphCast team at Google introduced GraphCast. This model was able to outperform the industry standard High Resolution Forecast (HRES) produced by the European Centre for Medium-Range Weather Forecasts (ECMWF). This model as the name suggests leverages graphing connections rather than traditional grid like data structure. The base data is given in latitude and longitude degrees at a resolution of 0.25 degrees. This means points are closer to each other at the poles. Using the graphing structure is supposed to help with bias incurred as a result of this\cite{GraphCast}.

There has been a lot of progress made over the last 6 years (since 2018) and especially in the last 2 years (since 2022)\cite{SecondRevolution}. The progression from machine learning methods being an interesting idea in the field of numerical weather predictions, to outperforming the standard NWP has been remarkably quick. Two years ago, machine learning methods were able to predict quickly and in some niche cases outperform traditional models. They were not generally competitive with standard weather models. Now they are competitive. It is worth noting that the training of these large models is based on data from traditional large weather models. It will be very interesting to watch what the next few years will have in store for the development of machine learning in weather predictions.

\section{Methodology and related work}
This study looks at data from three sources and an attempt is made to predict the gust factor in a given place in Iceland. It uses reanalysis data, along with elevation data to predict the gust factor. It looks at the data at the point of interest. It does not look at the data as a time series. This thesis aims to improve on the baseline model of always predicting the mean gust factor and show that some structure can be learned from reanalysis data about gusts. To do this a neural network was created. Any significant improvement on a base model, that always guesses the mean gust factor, would indicate that the final model has something to contribute.

In 2004, H. Ágústsson and H. Ólafsson\cite{mean_gust_HA_HO} looked at the variability of gust factor in complex landscapes. They looked at data from automatic weather stations that measure wind at 10 meters above ground. The data that was studied in 2004 comes from the same source as used in this thesis, but limits itself to a smaller section. They only looked at the years 1999-2001. They looked at three factors and how these three parameters effected the gust factor. These were $d_m, D, H$, that is direction of wind blowing off a mountain, distance to the mountain and the height of the mountain above the weather station. Their main results were that the gust factor is inversely correlated to the distance from a mountain and correlated to the height of the mountain. The study in 2004 looked at the effect of a dominant point upwind. It did not look at the effects of the landscape more broadly. In this study, landscape upwind is looked at.

\subsection{Model architechture}
To be able to capture the patterns in the data a neural network was constructed. A NN architechture was chosen as they are known to be able to capture patterns well in complex data and handle high parameter counts. This comes in handy when training on different types of data. It is also easy to construct different types of neural networks and see how they fit well with parts of the dataset. To measure the performance of these models, both to train and test, mean absolute percentage error (MAPE) as defined in Equation (\ref{eqn:mape}) was used.
\begin{equation}
    \label{eqn:mape}
    \text{MAPE} = \frac{1}{n}\Sigma_{i=1}^n\frac{|y_{predict} - y_{true}|}{y_{predict}}
\end{equation}
This was chosen because the target is the gust factor (the wind gust over the average wind). If the target would have been the wind gust rather than the gust factor then something like mean absolute error might be more appropriate.

\subsection{Model explainability}
Neural networks are often considered as mysterious black boxes\cite{nn_black_box}. In an attempt to understand the model predictions, methods designed for explainability are used. One such method is Shapley values\cite{shapley_information}. Shapley values are calculated as the average marginal contribution of a feature value across all possible coalitions. For any combination of parameters what is the contribution of a given parameter. This means that Shapley values can explain individual predictions. Other machine learning tools, like ELI5 (Explain like I am 5), randomly shuffle a feature and look at the effect on model performance\cite{eli5_information}.
% Chapter Template

\chapter{Data gathering and processing}
\label{Chapter2}
Data was sourced from several streams. The Icelandic Meteorological Office (IMO) provided measurements from weather stations all around Iceland. NWP data was downloaded from Copernicus Arctic Regional Reanalysis dataset (CARRA). A land elevation model was also provided by IMO.

\section{Automatic Weather Station Data}

IMO provided 10 minute measurements from \nStationsMin weather stations all around Iceland. The measurements that met the filtering criteria, started in \startDateVedur and ended in 2023. Of these \nStationsMin stations, \nVedurMin were from IMO and placed at 10 meters above ground, while the rest (\nVGMin) were from \href{https://www.vegagerdin.is/}{the Icelandic Road and Coastal Administration (IRCA)} and placed at 6-7 meters above ground\cite{vegagerdin_postur}. The location of these weather stations can be seen in Figure \ref{fig:aws_map}. The information that is provided by these Automatic Weather Stations (AWS) is presented in two different types of data files, hourly and 10 minute files. The hourly files are summations of the 10 minute files, with the exception that errors, such as nails, still in the 10 minute files should have been removed from the hourly documents. Nails, are sharp increases from the rest of the data and are unrealistic outliers that are considered measurement errors and are discarded. Each type of document contain the following information: the date and time, the station number (that can be converted to the coordinates using another data set), the average wind speed ($f$), the wind gust ($f_g$), the standard deviation of the wind gust, the direction of the wind ($d$) and the standard deviation for the wind direction. These measurement started at the end of the 20th century, when the first AWS stations was installed. More have been added in the following decades. This thesis does not look at the data as a time series, it tries to make predictions using only the information at a given point in time.

\begin{figure}
    \centering
    \includegraphics[scale = 1]{Figures/stationsOverIceland_2024-05-16_stripped_to_frame.png}
    \caption[Locations of automatic weather stations in Iceland]{Locations of all 412 stations that were looked at in this study. Most of these were from IMO but over a hundred were from IRCA. IMO AWS are placed at 10 meters above ground, while IRCA stations are placed at around 6-7 meters above ground.}
    \label{fig:aws_map}
\end{figure}

\section{CARRA Data}
The CARRA dataset goes back to September 1991 and is currently updated monthly, with a latency of 2-3 months\cite{carra_information}. The oldest IMO data point that fulfills given criteria is from \startDateVedur. This is covered by CARRA. The CARRA dataset is available for two regions, West and East. Each of these covers a vastly larger area than the area of interest. This leads to having to store a large amount of data. To get the data one has two options. Their web interface or using their API client. Using the API client is the only realistic option here, as there were thousands of requests made for different times. If using the API, it is possible to query a smaller area (such as a rectangular area around Iceland) given a set of coordinates.

The requests to the API are made at each available CARRA hour ([00, 03, 06, 09, 12, 15, 18, 21]) for each available observation. Only these hours can be used as the CARRA predicted output is represents the wind speed at those given hours and would thus not encapsulate well the extreme values in between these times. Using these datapoints, interpolation was used to get an estimation for the point of the given weather station. The CARRA data contains several types of layers. These are single levels, model levels, height levels, pressure levels. The data for this thesis was downloaded from height levels. That is, data was requested at heights of 15, 250 and 500 meters above ground. For each point 4 parameters were requested, wind speed, wind direction, pressure and temperature. Each of these features needed to be interpolated to create data for model to be trained on.

\section{Elevation data}
IMO provided a TIFF file containing the elevation of Iceland on a 20 meter by 20 meter grid. This file encompasses Iceland and is around 685 MB. The Python package Rasterio allowed for quick lookup with it's index and the affine transform. Using this package it is possible to quickly look up elevation given coordinates using matrix calculations. This package allowed for lookup by coordinates and by positioning inside the grid encompassing Iceland. It also allows for index lookup using coordinates. Using index lookup, neighboring points can be looked up and used to bridge the elevation for exact posisiton of given point.

\label{Chapter3} 
\section{Combining data sources}

This project used three main data sources, which need to be queried, filtered and combined to prepare the data for use in the models. When working with hundred of thousands of rows, the efficiency of the code is very important. Iterating through those rows might be necessary at times but will increase the time exponentially as compared to using vectorizing methods were possible. The three data sources were all in a different format. Measurement data from IMO was in text files, elevation data was in GeoTiff and reanalysis data from CARRA was in a GRIB format. To use the data to train, these three data sources needed to be combined into one file. This was done based on the measurement data from the IMO. A limit was set on the average wind speed and it was used to select measurement points. Along with the average wind speed having to be above a certain limit, to ensure that the same weather for the same location is not duplicated. The data from IMO was supplied for 10 minute increments, while CARRA data is in 3 hour intervals. This means that to use the CARRA data to predict the measured values from IMO, temporal interpolation would need to be done. Along with the temporal interpolation, note that the CARRA data is given in a rectangular grid where the distance between each point is around 2.5 km while the the information from the IMO is given at specific locations. The elevation information was given by a 20 by 20 rectangular grid that covers Iceland. When combining these data sources an interpolation method needs to be decided upon. Here linear interpolation was applied, both temporally and spatially. As the measurement points chosen were not outliers, that is the strongest average wind speed in a given range, interpolating in time does not work well. This necessitates using only measurements that are exactly at three hour intervals (00, 03, 06, 09, 12, 15, 18, 21 hours) at interpolating only in space. The choice of interpolation method, although potentially impactful on the results, was not specifically addressed in this study.

\begin{figure}[h]
    \includegraphics[scale = 0.5]{Figures/data-preprocessing-flow-chart.png}
    \caption{A flow chart showing how data sources were combined}
    \label{fig:data_preprocessing_flow_chart}
\end{figure}

The procedure of combining these sources was as follows and can be seen in Figure \ref{fig:data_preprocessing_flow_chart}. The measured data from the AWS is filtered by using a limit on the average wind speed. The gust factor generally drops with increased wind speed (although not always dependent on factors such as the landscape \cite{GNP_vidtal}). Even so being able to predict the gust factor is more important for higher average wind speed due to higher wind gusts. After this stripped dataset over every AWS has been created it is used to query the CARRA data by using their API. The CARRA API needs to be queried for given hours, days, months, years and a given area. That is, if queried for a given hour, it returns that hour for every day that is queried. Similarly if queried for a given day, it returns that day for every month. In light of these restraints, it was decided to query month by month. Querying only the days needed but every hour of the day (UTC 00, 03, 06, 09, 12, 15, 18 and 21). After querying and downloading the data for the height levels and variables requested, points of interest are interpolated and values stored in a pandas dataframe. After this the downloaded data is discarded and the next month is queried. This drastically decreases the amount of data that needs to be stored as compared to downloading the entire area and keeping all the data points (a reduction from several terabytes to less than a gigabyte).

Once CARRA data has been merged with AWS data, using station and time columns, then this combined file needs to checked for nails. This is done by using the hourly data (which is supposedly error free). As the data has been filtered in for the highest average wind speed in a 48 hour interval, the hourly data can be used to find nails. The hourly data is combined with the merged AWS and 10 min data. Then filtering is applied on the average wind. If the average wind speed for the 10 minute data is higher than the hourly the rows are dropped. Most stations have nails in less that 10\% measurements. The stations that have higher than 10\% error are ignored.

The elevation data comes in a GeoTIFF file that covers Iceland. It is a rectangular grid of resolution 20 meters. For every point of interest (every weather station), the elevation of that given point along with other points surrounding the weather station is retrieved. For each point retrieved interpolation needs to be done. This is done in a similar manner to the interpolation of the CARRA data. The four points bounding the point of interest were used to linearly interpolate the value of the point of interest. This information is included in the training data as the landscape is known to influence both the average wind and the gustiness \cite{GNP_vidtal}.

The error in reanalysis wind speed and measured wind speed can be significant. The absolute error increases as the measured wind speed increases, while the percentage wind speed decreases. A grouping of these errors by wind speed can be seen in Table \ref{table:measuredVSReanalysis_wind_speed}


\begin{table}[h]
    \caption[Comparison of measured and reanalysis wind speed]{Comparison of measured and reanalysis wind speed using mean absolute error (MAE) and mean absolute percentage error (MAPE). Note that for the computation of MAPE for ranges that otherwise include 0, 0 values have been excluded so as to prevent division by zero and exploding values. The comparisons are done using measured wind speed (at 10 meters above ground for IMO and 6-7 meters above ground for IRCA) and reanalysis wind speed at 15 meters above ground.}
    \label{table:measuredVSReanalysis_wind_speed}
    \centering
    \begin{tabular}{ccc}%c}
        \toprule
        f & n & MAE \\%& MAPE \\
        \midrule
        $[0;5[$ & 6.2e6 & 2.1 \\%& 1.6\\
        $[5;10[$ & 4.2e6 & 2.2 \\%& 0.3\\
        $[10;15[$ & 1.5e6 & 2.5 \\%& 0.2\\
        $[15;20[$ & 3.9e5 & 3.0 \\%& 0.2\\
        $[20;25[$ & 8.4e4 & 4.0 \\%& 0.2\\
        $[25;\infty[$ & 2.0e4 & 6.6 \\%& 0.2\\
        $[0;\infty[$ & 1.2e7 & 2.2 \\%& 1.0\\
        \bottomrule
    \end{tabular}
\end{table}

Another thing to look at is the distribution of error by station, both in terms of their coordinates and number of measurements. Looking at Figure \ref{fig:station_mae_distribution}, this distribution can be seen.

\begin{figure}
    \includegraphics[scale=0.6]{Figures/MAEoverIceland.png}
    \caption[Distribution of mean absolute errors by station]{The distribution of mean absolute errors by station. Using mean absolute error instead of mean absolute percentage error allows for all points to be used. Mean absolute percentage error can only be used if 0 values are ignored.}
    \label{fig:station_mae_distribution}
\end{figure}

Table \ref{table:station_mae_distribution} shows the 5 best and worst stations in terms of MAE.

\begin{table}[h]
    \caption[Mean absolute difference of measured wind speed and reanalysis wind speed at select stations]{Mean absolute error for reanalysis wind speed as compared to measured wind speed, for the five stations with the highest difference and the five stations with the lowest difference.}
    \label{table:station_mae_distribution}
    \resizebox{\textwidth}{!}{
    \centering
    \begin{tabular}{cccc}
        \toprule
        Station & Number of measurements & MAE & Location \\
        \midrule
        1470 & 6.8e3 & 1.17 & Reykjavík Háahlíð \\
        1350 & 5.2e4 & 1.18 & Keflavíkurflugvöllur \\
        1482 & 1.4e4 & 1.23 & Reykjavík Víðidalur \\
        4921 & 1.3e4 & 1.29 & Rif á Melrakkasléttu \\
        1477 & 5.6e4 & 1.29 & Reykjavíkurflugvöllur\\ \hdashline[0.5pt/1pt]
        35553 & 4.0e3 & 4.30 & Almannaskarð - göng\\
        6745 & 1.5e4 & 4.36 & Kerlingarfjöll - Ásgarðsfjall\\
        35978 & 7.9e3 & 4.40 & Fáskrúðsfjarðargöng suður\\
        2640 & 1.6e3 & 4.51 & Seljalandsdalur\\
        32635 & 3.2e4 & 4.95 & Botn í Súgandafirði\\
        \bottomrule
    \end{tabular}
    }
\end{table}

If the 0 m/s is excluded from the measurement data, then the error distribution using MAPE can be plotted. This plot can be seen in Figure \ref{fig:station_mape_distribution}

\section{Data Structure}

Once data has been retrieved for all three sources and processed, including interpolating values, it needs to be made ready to use by the model, for both training, validation and test. The starting point is a dataframe that contains measured information from AWS. This includes the average wind, the wind gust, wind direction along with the station number and coordinates. When selecting the CARRA data certain height levels are chosen. These present as separate lines in the CARRA dataframe. Information for one observation is represented in as many lines as height levels requested in the reanalysis data. These rows need to be combined on the position (the weather station). When this is done it is possible to combine the AWS IMO data and CARRA reanalysis data on the location and time columns. The last data source is the elevation. A circle sector upwind is looked at. In any case the points, that represent these sections, were selected as shown in Code Listing \ref{code:sectorElevation}. A range of angles are defined based on the wind direction $d$ at some distance from the given point. This means that the resultant points (equal in number to the length of angleRange by k) from arcs at several distances from the given weather station.

\begin{lstlisting}[style = Python, caption = {Sector elevation points generated}, label = code:sectorElevation]
angles = [(angle + (90 - d)) * pi/180 for angle in angleRange]
length_rng = [(exp(i * log(n + 1)/ k) - 1) * 1000 
                for i in range(1, k + 1)]
points = np.array([[(X + l * cos(angle), Y + l * sin(angle))
                    for angle in angles] for l in length_rng])   
\end{lstlisting}

The result is a dataframe that has measured data from AWS, which gives us our target, reanalysis data from CARRA, which gives us weather variables to train on, and finally elevation points in the landscape to include in our training data. An example of what the data looks like can be seen in Table \ref{table:trainDataExample}.

\begin{table}[h]
    \caption[An example of data structure used to train model]{An example of data structure used to train model. Data points include the derived variables Ri and N, the elevation of the station, direction of wind and relative direction of the wind (twd, that is the direction of the wind relative to center of Iceland), along with some combination of wind speed, pressure and temperature at the different height levels. Finally there are the elevation points around a given station, where the elevation is relative to the station.}
    \label{table:trainDataExample}
    \resizebox{\textwidth}{!}{
    \centering
    \begin{tabular}{cccccccccc}
        \toprule
        Ri & $N^2$ & station elevation & twd & $ws_{15}$ & $wd_{15}$ & $t_{15}$ & $p_{15}$ & $elevation_0$ & \dots\\
        \midrule 
        -1.18e+00 &  2.67e+04 & 100 & 1.5 & 10 & 5 & 0 & 100 & 2 & \dots\\
        \bottomrule
    \end{tabular}
    }
\end{table}

Looking at Table \ref{table:trainDataExample} note that the first two columns represent two variables that describe the stability of the air. These are the Richardson number ($Ri$)\cite{richardson_number_skybrary} and Brunt–Väisälä frequency\cite{brunt_vaisala_freq_eumtrain} ($N$), and are calculated using Equations (\ref{eqn:Ri}) and (\ref{eqn:N})\cite{mean_gust_HA_HO}. These values are calculated using reanalysis data at two different height levels. Thus $Ri$ refers to the Richardson number calculated between height levels 15m and 500m. Exactly the same notation is used with the Brunt–Väisälä frequency, except the square is used.

\begin{equation}
    \label{eqn:Ri}
    Ri = \frac{g \cdot dT \cdot dz}{T_{\textrm ave} \cdot dU^2} \unit{[]}
\end{equation}

\begin{equation}
    \label{eqn:N}
    N = \sqrt{\frac{g \cdot dT }{T_{\textrm ave} \cdot dz}} \unit{[Hz]}
\end{equation}

Here, $g$ is the acceleration due to gravity, $dT$ is the temperature difference between the two height levels, $dz$ is the elevation difference, $T_{\textrm ave}$ is the average temperature (that is the average of the two temperatures in the height levels) and $dU$ is the wind speed difference between the two height levels. Both of these numbers provide some insight about the stability of the air. A lower value for the Richardson number indicates a higher turbulence. A typical range of values could be between 0.1 and 10, with values below 1 indicating significant turbulence\cite{richardson_number_skybrary}. When the square of the Brunt-Väisälä frequency is negative, then the air is unstable (an air parcel will move away from its original position)\cite{brunt_vaisala_freq_eumtrain}. These are derived factors from the reanalysis data and as such there shouldn't be a significant information gain using $Ri$ and $N$ as opposed to having the raw data. However, including these factors instead of every reanalysis variable requested might speed up training as well as making the model more easily explainable with the use of Shapley values or other tools for explainability. Using Shapley a feature importance value is attributed to a given feature by creating all possible permutations of any possible length (up to number of features) and seeing how the predictions are skewed when the given parameter is included or excluded. This needs to be done for all parameters. The time complexity of this is very high ($2^n$ coalitions)\cite{shapley_information}. Most implementations use some approximations, which still can take a considerable amount of time for models with a high parameter count and many examples. The Richardson number includes the difference in wind speeds in the denominator. In certain cases, where the difference in wind speed between two levels is very, can blow up to infinity. This can cause problems and distort the predicitons.

\section{Data distribution}

The CARRA data is reanalysis and as such might have a bias or some systemic distortion when compared to the measured data. The distribution of the observed and reanalysis wind speed can be seen in Figure \ref{fig:obs_carra_wind_speeds}. Looking at the figure, the reanalysis wind tends to be higher even though the distribution is similar. 

\begin{figure}[ht]
    \centering
    % First subfigure (top)
    \begin{subfigure}[b]{0.8\textwidth}
        \centering
        \includegraphics[width=\textwidth]{Figures/obs_wind_speeds.png}
        \caption[Histogram of observed wind speeds]{A histogram distribution of observed wind speeds provided by IMO and IRCA for all used weather stations.}
        \label{fig:obs_wind_speeds}
    \end{subfigure}
    
    \vspace{0.5cm} % Space between the two images

    % Second subfigure (bottom)
    \begin{subfigure}[b]{0.8\textwidth}
        \centering
        \includegraphics[width=\textwidth]{Figures/carra_wind_speeds.png}
        \caption[Distribution of interpolated CARRA wind speeds at weather stations]{Distribution of interpolated CARRA reanalysis values at weather stations. The reanalysis data is only available at 3 hour intervals and 2.5km grid. These values are interpolated from data given by CARRA at the weather stations.}
        \label{fig:carra_wind_speeds}
    \end{subfigure}

    \caption[Distribution of CARRA and observed wind speeds]{Distribution of CARRA reanalysis wind speed and observed wind speed at weather stations provided by}
    \label{fig:obs_carra_wind_speeds}
\end{figure}

\begin{figure}[ht]
    \centering
    \includegraphics[width=0.8\textwidth]{Figures/station_heights.png}
    \caption[Distribution of weather station heights above sea level]{Distribution of heights of weather stations above sea level. One station has been excluded as an outlier of well above 1000 meters, having few and inconsistent datapoints.}
    \label{fig:station_heights}
\end{figure}

\begin{figure}[ht]
    \centering
    \includegraphics[width=0.8\textwidth]{Figures/gust_factor_2025.png}
    \caption[Distribution of gust factors]{Histogram of gust factors. By definition the lower bound of gust factor is 1. The majority of observed gust factors fall in the range 1.2 to 2. Gust factor decreases with increasing wind speed.}
    \label{fig:gust_factors}
\end{figure}


%\chapter{Data processing and structure}
\label{Chapter3} 
\section{Combining data sources}

This project used three main data sources, which need to be queried, filtered and combined to prepare the data for use in the models. When working with hundred of thousands of rows, the efficiency of the code is very important. Iterating through those rows might be necessary at times but will increase the time exponentially as compared to using vectorizing methods were possible. The three data sources were all in a different format. Measurement data from IMO was in text files, elevation data was in GeoTiff and reanalysis data from CARRA was in a GRIB format. To use the data to train, these three data sources needed to be combined into one file. This was done based on the measurement data from the IMO. A limit was set on the average wind speed and it was used to select measurement points. Along with the average wind speed having to be above a certain limit, to ensure that the same weather for the same location is not duplicated. The data from IMO was supplied for 10 minute increments, while CARRA data is in 3 hour intervals. This means that to use the CARRA data to predict the measured values from IMO, temporal interpolation would need to be done. Along with the temporal interpolation, note that the CARRA data is given in a rectangular grid where the distance between each point is around 2.5 km while the the information from the IMO is given at specific locations. The elevation information was given by a 20 by 20 rectangular grid that covers Iceland. When combining these data sources an interpolation method needs to be decided upon. Here linear interpolation was applied, both temporally and spatially. The choice of interpolation method, although potentially impactful on the results, was not specifically addressed in this study.

\begin{figure}[h]
    \includegraphics[scale = 0.5]{Figures/data-preprocessing-flow-chart.png}
    \caption{A flow chart showing how data sources were combined}
    \label{fig:data_preprocessing_flow_chart}
\end{figure}

The procedure of combining these sources was as follows and can be seen in Figure \ref{fig:data_preprocessing_flow_chart}. The measured data from the AWS is filtered by using a limit on the average wind speed. The gust factor generally drops with increased wind speed (although not always dependent on factors such as the landscape \cite{GNP_vidtal}). Even so being able to predict the gust factor is more important for higher average wind speed due to higher wind gusts. After this stripped dataset over every AWS has been created it is used to query the CARRA data by using their API. The CARRA API needs to be queried for given hours, days, months, years and a given area. That is, if queried for a given hour, it returns that hour for every day that is queried. Similarly if queried for a given day, it returns that day for every month. In light of these restraints, it was decided to query month by month. Querying only the days needed but every hour of the day (UTC 00, 03, 06, 09, 12, 15, 18 and 21). After querying and downloading the data for the height levels and variables requested, points of interest are interpolated and values stored in a pandas dataframe. After this the downloaded data is discarded and the next month is queried. This drastically decreases the amount of data that needs to be stored as compared to downloading the entire area and keeping all the data points (a reduction from several terabytes to less than a gigabyte).

Once CARRA data has been merged with AWS data, using station and time columns, then this combined file needs to checked for nails. This is done by using the hourly data (which is supposedly error free). As the data has been filtered in for the highest average wind speed in a 48 hour interval, the hourly data can be used to find nails. The hourly data is combined with the merged AWS and 10 min data. Then filtering is applied on the average wind. If the average wind speed differs the rows are dropped.

After filtering out nails, most stations have nails in less that 10\% measurements. The stations that have higher than 10\% error are ignored.

The elevation data comes in a GeoTIFF file that covers Iceland. It is a rectangular grid of resolution 20 meters. For every point of interest (every weather station), the elevation of that given point along with other points surrounding the weather station is retrieved. For each point retrieved interpolation needs to be done. This is done in a similar manner to the interpolation of the CARRA data. The four points bounding the point of interest were used to linearly interpolate the value of the point of interest. This information is included in the training data as the landscape is known to influence both the average wind and the gustiness \cite{GNP_vidtal}.

The error in reanalysis wind speed and measured wind speed can be significant. The absolute error increases as the measured wind speed increases, while the percentage wind speed decreases. A grouping of these errors by wind speed can be seen in Table \ref{table:measuredVSReanalysis_wind_speed}


\begin{table}[h]
    \caption[Comparison of measured and reanalysis wind speed]{Comparison of measured and reanalysis wind speed using mean absolute error (MAE) and mean absolute percentage error (MAPE). Note that for the computation of MAPE for ranges that otherwise include 0, 0 values have been excluded so as to prevent division by zero and exploding values. The comparisons are done using measured wind speed (at 10 meters above ground for IMO and 6-7 meters above ground for IRCA) and reanalysis wind speed at 15 meters above ground.}
    \label{table:measuredVSReanalysis_wind_speed}
    \centering
    \begin{tabular}{cccc}
        
        f & n & MAE & MAPE \\
        
        {[}0;5{[} & 6.2e6 & 2.1 & 1.6\\
        {[}5;10{[} & 4.2e6 & 2.2 & 0.3\\
        {[}10;15{[} & 1.5e6 & 2.5 & 0.2\\
        {[}15;20{[} & 3.9e5 & 3.0 & 0.2\\
        {[}20;25{[} & 8.4e4 & 4.0 & 0.2\\
        {[}25;$\infty${[} & 2.0e4 & 6.6 & 0.2\\
        {[}0;$\infty${[} & 1.2e7 & 2.2 & 1.0\\
        
    \end{tabular}
\end{table}

Another thing to look at is the distribution of error by station, both in terms of their coordinates and number of measurements. Looking at Figure \ref{fig:station_mae_distribution}, this distribution can be seen.

\begin{figure}
    \includegraphics[scale=0.6]{Figures/MAEoverIceland.png}
    \caption[Distribution of mean absolute errors by station]{The distribution of mean absolute errors by station. Using mean absolute error instead of mean absolute percentage error allows for all points to be used. Mean absolute percentage error can only be used if 0 values are ignored.}
    \label{fig:station_mae_distribution}
\end{figure}

Table \ref{table:station_mae_distribution} shows the 5 best and worst stations in terms of MAE.

\begin{table}[h]
    \caption[Mean absolute difference of measured wind speed and reanalysis wind speed at select stations]{Mean absolute error for reanalysis wind speed as compared to measured wind speed, for the five stations with the highest difference and the five stations with the lowest difference.}
    \label{table:station_mae_distribution}
    \resizebox{\textwidth}{!}{
    \centering
    \begin{tabular}{cccc}
        
        Station & Number of measurements & MAE & Location \\
        
        1470 & 6.8e3 & 1.17 & Reykjavík Háahlíð \\
        1350 & 5.2e4 & 1.18 & Keflavíkurflugvöllur \\
        1482 & 1.4e4 & 1.23 & Reykjavík Víðidalur \\
        4921 & 1.3e4 & 1.29 & Rif á Melrakkasléttu \\
        1477 & 5.6e4 & 1.29 & Reykjavíkurflugvöllur\\ \hdashline[0.5pt/1pt]
        35553 & 4.0e3 & 4.30 & Almannaskarð - göng\\
        6745 & 1.5e4 & 4.36 & Kerlingarfjöll - Ásgarðsfjall\\
        35978 & 7.9e3 & 4.40 & Fáskrúðsfjarðargöng suður\\
        2640 & 1.6e3 & 4.51 & Seljalandsdalur\\
        32635 & 3.2e4 & 4.95 & Botn í Súgandafirði\\
        
    \end{tabular}
    }
\end{table}

If the 0 m/s is excluded from the measurement data, then the error distribution using MAPE can be plotted. This plot can be seen in Figure \ref{fig:station_mape_distribution}

\begin{figure}
    \includegraphics[scale=0.6]{Figures/MAPEoverIceland.png}
    \caption[Distribution of MAPE by station]{The distribution of MAPE. Using MAPE instead of MAE, necessitates the exclusion of 0 values from the ground truth (measured values).}
    \label{fig:station_mape_distribution}
\end{figure}

\section{Data Structure}

Once data has been retrieved for all three sources and processed, including interpolating values, it needs to be made ready to use by the model, for both training, validation and test. The starting point is a dataframe that contains measured information from AWS. This includes the average wind, the wind gust, wind direction along with the station number and coordinates. When selecting the CARRA data certain height levels are chosen. These present as separate lines in the CARRA dataframe. Information for one observation is represented in as many lines as height levels requested in the reanalysis data. These rows need to be combined on the position (the weather station). When this is done it is possible to combine the AWS IMO data and CARRA reanalysis data on the location and time columns. The last data source is the elevation. A sector of a circle looking upwind is looked at. In any case the points, that represent these sections, were selected as shown in Code Listing \ref{code:sectorElevation}. A range of angles are defined based on the wind direction $d$ at some distance from the given point. This means that the resultant points (equal in number to the length of angleRange by k) form sectors at several distances from the given weather station.

\begin{lstlisting}[style = Python, caption = {Sector elevation points generated}, label = code:sectorElevation]
angles = [(angle + (90 - d)) * pi/180 for angle in angleRange]
length_rng = [(exp(i * log(n + 1)/ k) - 1) * 1000 
                for i in range(1, k + 1)]
points = np.array([[(X + l * cos(angle), Y + l * sin(angle))
                    for angle in angles] for l in length_rng])   
\end{lstlisting}

The result is a dataframe that has measured data from AWS, which gives us our target, reanalysis data from CARRA, which gives us weather variables to train on, and finally elevation points in the landscape to include in our training data. An example of what the data looks like can be seen in Table \ref{table:trainDataExample}.

\begin{table}[h]
    \caption[An example of data structure used to train model]{An example of data structure used to train model. Data points include the derived variables Ri and N, the elevation of the station, direction of wind and relative direction of the wind (twd, that is the direction of the wind relative to center of Iceland), along with some combination of wind speed, pressure and temperature at the different height levels. Finally there are the elevation points around a given station, where the elevation is relative to the station.}
    \label{table:trainDataExample}
    \resizebox{\textwidth}{!}{
    \centering
    \begin{tabular}{c|c|c|c|c|c|c|c|c|c}
         Ri & N\_squared & station\_elevation & twd & ws\_15 & wd\_15 & t\_15 & p\_15 & elevation\_point\_0 & \dots\\\hline
         -1.18e+00 &  2.67e+04 & 100 & 1.5 & 10 & 5 & 0 & 100 & 2 & \dots 
    \end{tabular}
    }
\end{table}

 Looking at Table \ref{table:trainDataExample} note that the first two columns represent two variables that describe the stability of the air. These are the Richardson number ($Ri$)\cite{richardson_number_skybrary} and Brunt–Väisälä frequency\cite{brunt_vaisala_freq_eumtrain} ($N$), and are calculated using Equations (\ref{eqn:Ri}) and (\ref{eqn:N})\cite{mean_gust_HA_HO}. These values are calculated using reanalysis data at two different height levels. Thus $Ri$ refers to the Richardson number calculated between height levels 15m and 500m. Exactly the same notation is used with the Brunt–Väisälä frequency, except the square is used.
\begin{equation}
    \label{eqn:Ri}
    Ri = \frac{g \cdot dT \cdot dz}{T_{\textrm ave} \cdot dU^2} \unit{[]}
\end{equation}

\begin{equation}
    \label{eqn:N}
    N = \sqrt{\frac{g \cdot dT }{T_{\textrm ave} \cdot dz}} \unit{[Hz]}
\end{equation}

Here, $g$ is the acceleration due to gravity, $dT$ is the temperature difference between the two height levels, $dz$ is the elevation difference, $T_{\textrm ave}$ is the average temperature (that is the average of the two temperatures in the height levels) and $dU$ is the wind speed difference between the two height levels. Both of these numbers provide some insight about the stability of the air. A lower value for the Richardson number indicates a higher turbulence. A typical range of values could be between 0.1 and 10, with values below 1 indicating significant turbulence\cite{richardson_number_skybrary}. When the square of the Brunt-Väisälä frequency is negative, then the air is unstable (an air parcel will move away from its original position)\cite{brunt_vaisala_freq_eumtrain}. These are derived factors from the reanalysis data and as such there shouldn't be a significant information gain using $Ri$ and $N$ as opposed to having the raw data. However, including these factors instead of every reanalysis variable requested might speed up training as well as making the model more easily explainable with the use of Shapley values or other tools for explainability. Using Shapley a feature importance value is attributed to a given feature by creating all possible permutations of any possible length (up to number of features) and seeing how the predictions are skewed when the given parameter is included or excluded. This needs to be done for all parameters. The time complexity of this is very high ($2^n$ coalitions)\cite{shapley_information}. Most implementations use some approximations, which still can take a considerable amount of time for models with a high parameter count and many examples.
\chapter{Discussion and conclusions}
\label{Chapter5}

\section{Main findings}
As shown in Table \ref{table:setsOfParams}, the neural-network model achieves a significant reduction in mean absolute percentage error (MAPE) compared to a constant baseline as well as a simple regression model. Table \ref{table:results} further demonstrates this improvement. For reanalysis wind speeds above 10~m/s, the network reduces error by about 30\% relative to the regression baseline. Table \ref{table:closed_intervals} shows the results of training and model evaluation within closed intervals of the reanalysis wind speed. The best results appear in the 10--25~m/s range. At lower reanalysis wind speeds, percentage errors inflate even when the differences between wind gust and average wind speed remain small and at both at the higher and the lower speeds the base line MAPE is higher.

Although one might expect staility indicators, such as the Richardson number and the Brunt--Väisälä frequency, to be among the strongest predictors of gust variability, they contributed less than other variables. The influence of individual features on model performance is illustrated by Shapley plots (Figures \ref{fig:ShapleySummary}, \ref{fig:ShapleySummaryKeflavikurflugvollur}, \ref{fig:ShapleySummaryAlmannaskarð}, \ref{fig:ShapleySummaryAkrafjall} and figures in Appendix \ref{appendix:A}). According to the Shapley analysis, stations elevation, mean wind speed and wind direction are among the more important features for predicting the gust factor, and also, to some extent, the Brunt--Väisälä frequency ($N^2$). As explained in Section \ref{sec:shapley} it was infeasible to evaluate the DEM data with Shapley values.

Predictive skill varies significantly from station to station. This variability can be partly attributed to discrepancies between reanalysis and observations (Table \ref{table:station_mae_distribution}), differences in local terrain, and the number of measurements per site. Training separate models for individual stations did not improve performance (Tables \ref{table:specific_sites} and \ref{table:more_specific_sites}); in fact, some well-sampled stations performed worse when modeled in isolation. Future work should investigate whether these anomalies arise from systematic measurement errors, inherent atmospheric variability at those sites, or other factors.

\section{Limitations and future work}
This study did not include station coordinates (X, Y) as direct inputs. Adding location information could improve site-specific accuracy without losing the benefits of training on the full dataset. Similarly, instead of applying a fixed linear interpolation to CARRA and DEM grids, the NN training could receive raw neighboring grid values and their relative positions as part of their input, allowing the model to learn an optimal interpolation scheme.

Finally, treating the data as a time series may offer further gains. Here, only instantaneous inputs at forecast time were used, partly because CARRA analysis fields are available every three hours and may not capture rapid stability changes. Access to higher-frequency data or the inclusion of temporal history could enable models to exploit evolving atmospheric conditions and further enhance gust-prediction skills.

Many papers cited in this thesis suggest that adding a machine learning model as a backend to more traditional numerical atmospheric models could lead to improved performance. The results of this study support this.

\section{Possible applications}
Wind gusts at weather stations are displayed on road signs in Iceland, when wind gusts exceed \href{https://www.vegagerdin.is/vegagerdin/starfsemi/frettir/tvo-ny-vedur-og-upplysingaskilti-i-notkun#}{certain values}. The model could be used to allow such predicitons of wind gusts to be displayed for every section of the road, not just at the stations. In 2023, outside of Reykjavik harbor, a cruise ship nearly ran aground due to an unexpected wind gust \parencite{strand_skemmtiferdaskips}. This wind gust of \textasciitilde 26 m/s, pushed the ship into a buoy near the island Viðey. It was only luck that prevented the ship's screws from hitting the buoy. The ship was only 10 meters from the shoals of Viðey, where the water is only 0.4 meters deep. A better prediction of wind gust could have prevented this incidence. Similarly, the model could be used to prevent damage to transportation vehicles, that take on a lot of wind, such as trucks and buses, or to prevent damage to buildings.

The power generated by wind turbines increases with the cube of the wind speed \cite{wind_power}. The highest wind gusts in Iceland are around 70 m/s. Knowing the gust factor accurately would allow the turbine to be stopped in time, thus avoiding damage.

\section{Conclusions}
The neural network model presented in this study demonstrates a significant improvement in predicting wind gusts over a simple regression model, particularly for reanalysis wind speeds above 10 m/s. The model's performance varies by station, and while some stations benefit from the full dataset, others may require more localized training. 
\chapter{Results}
\label{Chapter5}
\section{Results}
A baseline model was constructed. This model looked at the gust factor for some training data and took the average of this and predicted this average everytime. Several baseline guesses were created based on a lower limit assigned to the average wind speed limit (AWSL). As the average wind speed increases, then the variability in gust as a percentage decreases\cite{mean_gust_HA_HO}. This means looking at a subset of the data where the AWSL is higher, a better result can be expected. The results show this. Several different AWSL were used for the baseline model, as can be seen in table \ref{table:results}. This sets a goal. A model that does not significantly improve on this baseline suggests either failure to capture essential patterns in the data or that the data itself may lack the necessary information for substantial improvements upon the baseline. Using the previously described neural network architecture setups for each AWSL, with and without landscape elevation information, MAPE was determined. The results can be seen in Table \ref{table:results}.

\begin{table}[h]
    \caption[Model results for different AWSL]{MAPE for each average wind speed limit with and without landscape elevation in a 30° sector around the point of interest into the direction of the reanalysis wind. The influence of adding elevation data seems to reduce the error. The percentage error is higher for lower wind speeds and thus observing the error for different lower bound of wind speed will produce different results. This lower bound is determined using the reanalysis wind speed at 15 meters ($ws_{15}$).}
    \label{table:results}
    \centering
    \begin{tabular}{lccc}
        \toprule
        \textbf{AWSL} & \textbf{MAPE} & &\\ 
        $[m/s]$ & \textit{Baseline} &  \textit{Without DEM} & \textit{With DEM} \\\hline
        $\geq 0$ & 39.2\% & 19.7\% & 18.6\% \\
        $\geq 5$ & 28.1\% & 15.9\% & 14.1\%\\
        $\geq 10$ & 23.9\% & 13.6\% & 11.5\%\\
        $\geq 15$ & 23.2\% & 13.0\% & 10.6\%\\
        $\geq 20$ & 24.7\% & 13.6\% & 11.1\%\\
        $\geq 25$ & 27.7\% & 15.7\% & 12.9\%\\
        \bottomrule
    \end{tabular}
\end{table}

%m/s & $ws_{15}$ & f & $ws_{15}$ & f & $ws_{15}$ & f\\\hline
%$\geq 0$ & 39.2\% & 39.2\% & 19.7\% & 15.9\% & 18.6\% & 15.3\%\\
%$\geq 5$ & 28.1\% & 14.8\% & 15.9\% & 9.9\% & 14.1\% & 9.4\%\\
%$\geq 10$ & 23.9\% & 11.1\%& 13.6\% & 7.6\% & 11.5\% & 7.4\%\\
%$\geq 15$ & 23.2\% & 9.3\% & 13.0\% & 7.0\% & 10.6\% & 6.4\%\\
%$\geq 20$ & 24.7\% & 8.2\% & 13.6\% & 6.3\% & 11.1\% & 5.8\%\\
%$\geq 25$ & 27.7\% & 7.3\% & 15.7\% & 5.8\% & 12.9\% & 5.5\% \\
%7.5 & 25.3\% & 12.6\% & 14.5\% & 8.6\% & 12.5\% & 8.6\%\\
%12.5 & 23.3\% & 10.1\% & 13.1\% & 7.1\% & 11.0\% & 6.9\%\\
%17.5 & 23.7\% & 8.7\% & 12.8\% & 6.4\% & 10.5\% & 6.1\%\\
%22.5 & 26.1\% & 7.8\% & 14.5\% & 6.1\% & 12.6\% & 5.7\%\\

This is some improvement upon the baseline error, with a decrease in error from 23.9\% to 13.6\% and 10.6\% for the baseline, model without DEM and with model with DEM for 10 m/s cutoff. The power generated by wind mill increases with wind speed cubed\cite{wind_power}. The highest wind gusts in Iceland are around 70 m/s. Knowing the gust factor with half as much error as before can allow better anticipation and thus spare turbines for high wind gusts. Another way to look at the error improvement is by station. No location data was directly included in the training data. In Table \ref{table:station_mae_distribution} the mean absolute error of predicted average wind speed and measured average wind speed can be seen for the extreme values. A question to ask is it possible to achieve better results when only looking at a single station?

\begin{table}[h]
    \caption[Model result by station]{The MAPE results for selected stations of interest, both when training for the specific site and when the stations are a part of the general data. For every station the AWSL is set at 10 m/s. In training for a single station at a time, some site specific information can be gauged. This does not mean that the a better result can be reached for that site. Factors such as the number of datapoints at given location can significantly impact the result. This table uses the measured wind speed to determine the cut off for data points. This leads to some data leakage and an increased performance compared to using the reanalysis CARRA speed for cut off.}
    \label{table:specific_sites}
    \centering
    \begin{tabular}{lccc}
        \textbf{Station name} & \textbf{Number of measurement points} & \multicolumn{2}{c}{\textbf{MAPE}}\\
        & & General training & Site training\\\hline
        Akrafjall & 42,791 & 18.6\% & 93.7\%\\
        Almannaskarð & 4,014 & 12.2\% & 86.7\%\\
        Ásgarðsfjall & 15,121 & 9.1\% & 9.4\%\\
        Jökulheimar & 17,176 & 7.7\% & 7.7\%\\
        Sandbúðir & 18,718 & 6.8\% & 6.4\%\\
        Stórholt & 35,126 & 7.1\% & 29.2\% \\
        Þúfuver & 19,538 & 6.4\% & 6.8\%\\
    \end{tabular}
\end{table}

Instead of looking at the exact values of MAPE at select stations, a plot of the error distribution can be created. This can be seen in Figure (\ref{fig:errorMap}). Looking at Figure (\ref{fig:errorMap}), the worst performing stations can be seen at Vestfirðir and around the coastline. Stations further inland seem to have lower error. The worst performing station, at Seljalandsdalur, is at Vestfirðir while the best performing station Garðskagaviti is at the South-Western tip of Iceland in Reykjanes.

\begin{figure}[h]
    \centering
    \includegraphics[scale = 0.5]{Figures/errorMap.png}
    \caption[MAPE error distribution of stations shown on a map of Iceland.]{The MAPE error of each station in data shown as color gradient circles. That is each station is represented by one circle, with the error value represented as a color gradient from dark blue to yellow. The lowest error at a single station was around 7\% at Garðskagaviti and the highest around 35\% at Seljalandsdalur. It is important to note that the model is trained using a cutoff of 10 m/s and this cutoff point is determined using the reanalysis wind at 15 meters above ground.}
    \label{fig:errorMap}
\end{figure}

\begin{table}[h]
    \caption[Model result looking at closed wind speed intervals]{The MAPE results for different AWSL intervals. Here instead of training for all data above a certain threshold put of the wind, training is done only on data between two wind speeds. The percentage variance in gust factor as a function of wind speed increases with decreasing wind speed. Measured wind speed is used for the cutoff and thus have data leakage. This results should thus be somewhat comparable to the last column in Table (\ref{table:results})}
    \label{table:closed_intervals}
    \centering
    \begin{tabular}{l | cc}
        \textbf{Interval} &  \multicolumn{2}{c}{\textbf{MAPE}}\\
        $[m/s]$ & Without Elevation & With Elevation\\\hline
        $[5, 10[$ & 17.2\% & 15.4\%\\ %10.6\%\\
        $[10, 15[$ & 14.1\% & 11.9\%\\ %7.8\%\\
        $[15, 20[$ & 13.2\% & 10.9\%\\ %6.4\%\\
        $[20, 25[$ & 14.9 \% & 11.6\%\\ %6.3\%\\
        $[25, 30[$ & 17.0\% & 16.0\%\\ %7.1\%\\
    \end{tabular}
\end{table}

Table (\ref{table:closed_intervals}) shows no improvement over the values in the last column in Table (\ref{table:results}). As previously mentioned, the gust factor decreases with increasing wind speed and thus, training on intervals and lessening this effect might be expected to give better results. This does not seem to be the case. Some interesting sites to look closer at for drivers might include places like Kjalarnes, Ingólfsfjall, Þrengsli and others. These can be seen in Table (\ref{table:specific_sites}).

\begin{table}[h]
    \caption[Model result by stations of interest]{The MAPE results of different stations for several stations of interest, both when training for the specific site and when the stations are a part of the general data. For every station the AWSL is set at 10 m/s. In training for a single station at a time, some site specific information can be gauged. This does not mean that the a better result can be reached for that site. Factors such as the number of datapoints at given location can significantly impact the result. This table uses the measured wind speed to determine the cut off for data points. This leads to some data leakage and an increased performance compared to using the reanalysis CARRA speed for cut off.}
    \label{table:specific_sites}
    \centering
    \begin{tabular}{l | cc}
        \toprule
        \textbf{Station name} & \multicolumn{2}{c}{\textbf{MAPE}}\\
         & Baseline & Model\\
         \midrule
        Fáskrúðsfjörður & 28.2\% & 21.8\%\\
        Ingólfsfjall & 30.0\% & 19.6\%\\
        Kjalarnes & 20.7\% & 13.5\% \\
        Sandskeið & 13.0\% & 10.2\%\\
        Seyðisfjörður & 32.1\% & 23.0\%\\
        Þjórsárdalur & 12.2\% & 11.4\%\\
        Þrengsli & 13.6\% & 11.3\%\\
        \bottomrule
    \end{tabular}
\end{table}

\begin{figure}[h]
    \centering
    \includegraphics[scale = 0.6]{Figures/shap_plots/waterfall_plot.png}
    \caption[Feature importance for a single observation of a neural network.]{Feature importance of a neural network with model architecture as described in Table \ref{table:gridSearchHyperparamters} and data as described in Table \ref{table:trainDataExample}. In this specific instance the wind direction ($wd_{15}$) has the highest negative influence and the temperature has the highest positive influence. Elevation data is excluded when working with Shapley values, as the contribution of each elevation point is very low and there are very many of them. To see their influence on the model output see Table \ref{table:results}.}
    \label{fig:ShapleyWaterfall}
\end{figure}

In Figure (\ref{fig:ShapleySummary}) the contribution of each feature, excluding the elevation points, can be seen for the model in general (a significant subset of data is used). Looking at Figure (\ref{fig:ShapleySummary}), there is an outlier. Exactly calculating the Shapley values is time intensive, in the subset shown there is an outlier that skews the figure and makes it so that viewing the importance distribution excluding the outlier is difficult. For this reason, another shapley summary vas created that looked at different, and larger distribution. This can be seen in Figure (\ref{fig:ShapleySummary2}).

\begin{figure}
    \centering
    \includegraphics[scale = 0.6]{Figures/shap_plots/summary_plot.png}
    \caption[Summary feature importance of a neural network.]{Feature importance of a neural network with model architecture as described in Table \ref{table:gridSearchHyperparamters} and data as described in Table \ref{table:trainDataExample}. We can see that generally multiple factors influence the prediction, with the station elevation being highly influential. There is seemingly one outlier for the Richardson number, which usually has very little influence. Elevation data is excluded when working with Shapley values, as the contribution of each elevation point is very low and there are very many of them. To see their influence on the model output see Table \ref{table:results}.}
    \label{fig:ShapleySummary}
\end{figure}

\begin{figure}
    \centering
    \includegraphics[scale = 0.6]{Figures/shap_plots/summary_plot_190924_.png}
    \caption[Summary feature importance of a neural network using a larger distribution of data.]{Feature importance of a neural network with model architecture as described in Table \ref{table:gridSearchHyperparamters} and data as described in Table \ref{table:trainDataExample}. Generally multiple factors influence the prediction, with the station elevation being highly influential. Elevation data is excluded when working with Shapley values, as the contribution of each elevation. In contrast to Figure (\ref{fig:ShapleySummary}), the distribution doesn't have as extreme outliers. This means that more details can be seen in the figure. The X-axis shows the influence of feature values on the model. The color gradient shows the value of each feature. As an example, there is a very red value for station elevation (top line, all the way to the left). This means that in this instance, the station elevation contributed around -0.25 to the final output and that the station had an elevation significantly above average.}
    \label{fig:ShapleySummary2}
\end{figure}

The Figures (\ref{fig:ShapleySummary}) and (\ref{fig:ShapleySummary2}) show the summary for a model trained only on the features shown and not on landscape elevation of the surrounding area. This is done as the number of points there is too high to display in one figure (70 total points). Looking specifically at Figure (\ref{fig:ShapleySummary2}), the station elevation is most influential and the Richardson number's influence is very low. Most of the feature values bunch up in the middle, while the station elevation is elongated compared to other features. Station elevation seems to have two bunches on either side of 0. Overall, the values seem to be skewed to the right of 0. This would be expected as the predicted values are expected to be in the range of 1.2-2, or at least always above 1 by definition. Finally, for the Shapley values, looking that all the data there are again outliers that skew the data so that spotting general distribution is difficult. This can be seen in Figure (\ref{fig:ShapleySummary3}).

\begin{figure}
    \centering
    \includegraphics[scale = 0.6]{Figures/shap_plots/summary_plot_190924_full_10ms.png}
    \caption[Summary feature importance of a neural network using entire dataset.]{Feature importance of a neural network with model architecture as described in Table \ref{table:gridSearchHyperparamters} and data as described in Table \ref{table:trainDataExample}. The distribution seems to be the same as before, discounting the outliers.}
    \label{fig:ShapleySummary3}
\end{figure}

These plots can also be done by looking at individual stations. The model is then trained on all examples and afterwards the summary plot created using only points for each station. Stations of interest include stations listed in Table \ref{table:specific_sites}. These plots can all be seen in Figures (\ref{fig:ShapleySummaryAkrafjall} - \ref{fig:ShapleySummaryKeflavikurflugvollur}).

\begin{figure}
    \centering
    \includegraphics[scale = 0.6]{Figures/shap_plots/summary_plot_31572.png}
    \caption[Summary feature importance of a neural network only looking at AWS at Akrafjall.]{Feature importance of a neural network with model architecture as described in Table \ref{table:gridSearchHyperparamters} and data as described in Table \ref{table:trainDataExample}. This plot only looks at datapoints from Akrafjall. This seems to show the same distribution as previous summary plots. Station elevation is influential and Richardson number has no impact.}
    \label{fig:ShapleySummaryAkrafjall}
\end{figure}

\begin{figure}
    \centering
    \includegraphics[scale = 0.6]{Figures/shap_plots/summary_plot_35553.png}
    \caption[Summary feature importance of a neural network only looking at AWS at Almannaskarð.]{Feature importance of a neural network with model architecture as described in Table \ref{table:gridSearchHyperparamters} and data as described in Table \ref{table:trainDataExample}. This plot only looks at datapoints from Almannaskarð. This seems to show the same distribution as previous summary plots. Station elevation is influential and Richardson number has no impact.}
    \label{fig:ShapleySummaryAlmannaskarð}
\end{figure}

\begin{figure}
    \centering
    \includegraphics[scale = 0.6]{Figures/shap_plots/summary_plot_6745.png}
    \caption[Summary feature importance of a neural network only looking at AWS at Ásgarðsfjall.]{Feature importance of a neural network with model architecture as described in Table \ref{table:gridSearchHyperparamters} and data as described in Table \ref{table:trainDataExample}. This plot only looks at datapoints from Ásgarðsfjall. This seems to show the same distribution as previous summary plots. Station elevation is influential and Richardson number has no impact.}
    \label{fig:ShapleySummaryAsgarðsfjall}
\end{figure}

\begin{figure}
    \centering
    \includegraphics[scale = 0.6]{Figures/shap_plots/summary_plot_1470.png}
    \caption[Summary feature importance of a neural network only looking at AWS at Háahlíð.]{Feature importance of a neural network with model architecture as described in Table \ref{table:gridSearchHyperparamters} and data as described in Table \ref{table:trainDataExample}. This plot only looks at datapoints from Háahlíð. This seems to show the same distribution as previous summary plots. Station elevation is influential and Richardson number has no impact.}
    \label{fig:ShapleySummaryHaahlid}
\end{figure}

\begin{figure}
    \centering
    \includegraphics[scale = 0.6]{Figures/shap_plots/summary_plot_1350.png}
    \caption[Summary feature importance of a neural network only looking at AWS at Keflavíkurflugvöllur.]{Feature importance of a neural network with model architecture as described in Table \ref{table:gridSearchHyperparamters} and data as described in Table \ref{table:trainDataExample}. This plot only looks at datapoints from Keflavíkurflugvöllur. This seems to show the same distribution as previous summary plots. Station elevation is influential and Richardson number has no impact.}
    \label{fig:ShapleySummaryKeflavikurflugvollur}
\end{figure}

In each of these plots, even without the feature labels, the station elevation is easily noticed as the value is constant for a station. This is not noteworthy. What is noteworthy is the range of impact from this single value. For simpler models, this would not happen. It is important to note that SHAP assumes feature independece\cite{Salih_2024}. This might explain why the impact of the Richardson number is so low. Both the squared Brunt–Väisälä frequency and the Richardson number are derived features from reanalysis data. They carry with them some extra information over the other features in the dataset. This is because both are variables over elevation ranges. That is, as seen in Equations (\ref{eqn:Ri}, \ref{eqn:N}), both are dependent on values at lower and upper elevations and try to describe the stability of that range. Shapley tries to assign contribution values for each feature for each observation. SHAP assumes that the features are independent, but this is not the case. It is clearly not the case for the derived variables, but how the contribution should be distributed between the features is not clear. Seemingly the SHAP python package is giving all the impact to the Brunt–Väisälä squared frequency and none to Richardson number. If the Brunt–Väisälä would be excluded from the data, the impact of the Richardson number would likely increase. Another point to note is that the features are ordered by their impact. This means that the station elevation is the most impactful for each plot, but the ordering of other variables changes. Looking at Figure (\ref{fig:ShapleySummary3}), which shows the Shapley summary plot for all data, the wind speed is the second most important feature. This is reversed in Figure (\ref{fig:ShapleySummaryAkrafjall}). This is not unexpected as Akrafjall station was specifically selected as the MAE for reanalysis wind speed was very high as can be seen in Table (\ref{fig:ShapleySummaryAkrafjall}), where you will also find the stations whose summary plot is shown in Figures (\ref{fig:ShapleySummaryAlmannaskarð}, \ref{fig:ShapleySummaryAsgarðsfjall}) and these also fall into the category of very high MAE for wind speed. What is interesting is that the reanalysis wind speed is also of low impact at stations like Háahlíð and Keflavíkurflugvöllur, as shown in Figures (\ref{fig:ShapleySummaryHaahlid}, \ref{fig:ShapleySummaryKeflavikurflugvollur}). These stations had the lowest of MAE for difference between measured wind speed and reanalysis wind speed. As the summary plot over all stations (Figure (\ref{fig:ShapleySummary3})) shows that reanalysis wind speed is impactful, might lead to the conclusion that the reanalysis wind speed is not a good predictor at these locations or something else is skewing the data.

A simpler way to look at feature importance is to create models that are trained on and use to predict different sets of parameters. The results of such a comparison can be seen in Table \ref{table:setsOfParams}.

\begin{table}[h]
    \caption[Model results for different sets of parameters.]{...}
    \label{table:setsOfParams}
    \centering
    \begin{tabular}{lc}
        \toprule
        \textbf{Model parameters} & \textbf{MAPE}\\ 
        \midrule
        Baseline constant & 23.9\%\\
        $ws_{15}$ & 17.2\%\\
        $[ws_{15}, t_{15}, p_{15}, wd_{15}]$ 16.7\% & ---\\
        $[ws_{15}, t_{15}, p_{15}, wd_{15}, ASL, twd]$ & ---\\
        $[ws_{15}, t_{15}, p_{15}, wd_{15}, ASL, twd, N, Ri]$ & ---\\
        $[ws_{15}, t_{15}, p_{15}, wd_{15}, ASL, twd, N, Ri]$ + DEM & ---\\
        $[ws_{15,250,500}, t_{15,250,500}, p_{15,250,500}, wd_{15,250,500}, ASL, twd, N, Ri]$
        \bottomrule
    \end{tabular}
\end{table}
\chapter{Discussion}
\label{Chapter6}
The goal of this tesis was to research whether it was possible to use reanalysis data to increase the predictability of wind gusts and see what influence including the landscape elevation would have. A quick statistical analysis will give a p-value of 0, that is these results are statistically significant. The improvement shown in Table (\ref{table:results}) is not chance. There is some pattern to be learned in the interval that leads to a higher predictability of wind gusts over baseline.
\chapter*{Appendix A: Feature importance on Shapley plots}
\label{appendix:A}

\renewcommand{\thefigure}{A.\arabic{figure}} 
\setcounter{figure}{0} % Reset figure counter

\begin{figure}
    \centering
    \includegraphics[scale = 0.6]{Figures/shap_plots/summary_plot_190924_full_10ms.png}
    \caption[Summary feature importance of a neural network using entire dataset.]{Feature importance of a neural network with model architecture as described in Table \ref{table:gridSearchHyperparameters} and data as described in Table \ref{table:trainDataExample}. The distribution seems to be the same as before, discounting the outliers.}
    \label{fig:ShapleySummary3}
\end{figure}

\begin{figure}
    \centering
    \includegraphics[scale = 0.6]{Figures/shap_plots/summary_plot_6745.png}
    \caption[Summary feature importance of a neural network only looking at AWS at Ásgarðsfjall.]{Feature importance of a neural network with model architecture as described in Table \ref{table:gridSearchHyperparameters} and data as described in Table \ref{table:trainDataExample}. This plot only looks at datapoints from Ásgarðsfjall. This seems to show the same distribution as previous summary plots. Station elevation is influential and Richardson number has no impact.}
    \label{fig:ShapleySummaryAsgarðsfjall}
\end{figure}

\begin{figure}
    \centering
    \includegraphics[scale = 0.6]{Figures/shap_plots/summary_plot_1470.png}
    \caption[Summary feature importance of a neural network only looking at AWS at Háahlíð.]{Feature importance of a neural network with model architecture as described in Table \ref{table:gridSearchHyperparameters} and data as described in Table \ref{table:trainDataExample}. This plot only looks at datapoints from Háahlíð. This seems to show the same distribution as previous summary plots. Station elevation is influential and Richardson number has no impact.}
    \label{fig:ShapleySummaryHaahlid}
\end{figure}



\end{document}
