\chapter{Discussion and conclusions}
\label{Chapter5}
The goal of this tesis was to research whether it was possible to use reanalysis data to increase the predictability of wind gusts and see what influence including the landscape elevation would have. A quick statistical analysis will give a p-value of 0, that is these results are statistically significant. The improvement shown in Table (\ref{table:results}) is not chance. There is some pattern to be learned in the interval that leads to a higher predictability of wind gusts over baseline. There is also a large variability in the predictability based on weather stations. A large part of that can be attributed to the differing amount of data points for each station. It would be interesting to know how much of the predictability difference can be explained away with more data for these stations that have few data points or if there is some inherent difference. If there is inherent less predictability at those stations, what could be contributing factors in that?
