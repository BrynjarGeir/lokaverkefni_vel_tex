\chapter{Discussion and conclusions}
\label{Chapter5}

\section{Main findings}
As shown in Table \ref{table:results}, the neural-network model achieves a significant reduction in mean absolute percentage error (MAPE) compared to a constant baseline. Table \ref{table:setsOfParams} further demonstrates an improvement over a simple regression model. A statistical test yields $p<0.01$, confirming that these improvements are unlikely to be due to chance. The influence of individual features on model performance is illustrated by Shapley plots (Figure \ref{fig:ShapleySummaryKeflavikurflugvollur} and Appendix \ref{appendix:A}). Although one might expect stability indicators—such as the Richardson number and the Brunt--Väisälä frequency—to be among the strongest predictors of gust variability, they contributed less than other variables. For reanalysis wind speeds above 10~m/s, the network reduces error by about 30\% relative to the regression baseline. The largest gains appear in the 10--25~m/s range (Table \ref{table:closed_intervals}). At lower speeds, percentage errors inflate even when absolute gust--wind differences remain small; at higher speeds, sample sizes become sparse.

Predictive skill varies significantly from station to station. This variability can be partly attributed to discrepancies between reanalysis and observations (Table \ref{table:station_mae_distribution}), differences in local terrain, and the number of measurements per site. Training separate models for individual stations did not improve performance (Tables \ref{table:specific_sites} and \ref{table:more_specific_sites}); in fact, some well-sampled stations performed worse when modeled in isolation. Future work should investigate whether these anomalies arise from systematic measurement errors, inherent atmospheric variability at those sites, or other factors.

\section{Limitations and future work}
This study did not include station coordinates (XY) as direct inputs. Adding location information could improve site-specific accuracy without losing the benefits of training on the full dataset. Similarly, instead of applying a fixed linear interpolation to CARRA and elevation grids, the network could receive raw neighboring grid values and their relative positions, allowing it to learn an optimal interpolation scheme.

Finally, treating the data as a time series may offer further gains. Here, only instantaneous inputs at forecast time were used, partly because CARRA fields are available every three hours and may not capture rapid stability changes. Access to higher-frequency data or the inclusion of temporal history could enable models to exploit evolving atmospheric conditions and further enhance gust-prediction skill.

Many papers cited in this thesis suggest that adding a machine learning model as a backend to more traditional neural networks models could lead to improved performance. The results of this study support this.

\section{Possible applications}
Wind gusts are displayed on road signs in Iceland, if wind gusts exceed \href{https://www.vegagerdin.is/vegagerdin/starfsemi/frettir/tvo-ny-vedur-og-upplysingaskilti-i-notkun#}{expected values}. The model could be used to allow predicitons of wind gusts to be displayed. In 2023, outside of Reykjavik harbor, a cruise ship nearly \href{https://www.mbl.is/200milur/frettir/2024/04/26/strand_skemmtiferdaskips_kallar_a_erlenda_adstod/}{beached} due to an unexpected wind gust. This wind gust of \textasciitilde 26 m/s, pushed the ship into a buoy near Viðey. It was only luck that prevented the ship's screws from hitting the buoy. The ship was only 10 meters from the shoals of Viðey, where the water is only 0.4 meters deep. A better prediction of wind gust could have prevented this incident. Similarly, the model could be used to prevent damage to transportation vehicles, that take on a lot of wind, such as trucks and buses, or to prevent damage to buildings.

\section{Conclusions}
The neural network model presented in this study demonstrates a significant improvement in predicting wind gusts over a simple regression model, particularly for reanalysis wind speeds above 10 m/s. The model's performance varies by station, and while some stations benefit from the full dataset, others may require more localized training. The power generated by wind turbines increases with the cube of the wind speed \cite{wind_power}. The highest wind gusts in Iceland are around 70 m/s. Knowing the gust factor with half as much error as before can allow better anticipation and thus spare turbines for high wind gusts. 